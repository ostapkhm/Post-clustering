\documentclass[14pt]{article}

\usepackage[table]{xcolor}
\usepackage[a4paper, left=2cm, right=2cm, bottom=3cm]{geometry}
\usepackage[T1, T2A]{fontenc}
\usepackage[english, main=ukrainian]{babel}
\usepackage{tempora}
\usepackage{amssymb,amsfonts,amsmath,amsthm} 
\usepackage{dsfont}
\usepackage{empheq}

\begin{document}
\raggedright
\fontsize{12pt}{16pt}\selectfont

$(X, Y)-$ одне спостереження   
$K$ - фіксована кількість елементів суміші для $X$ \\
$P(\xi=i)=p_i, \: i= 1, \dots, K$ \\
$\xi$ вказує до якої суміші належить $X$ \\
$\eta-$ випадкова величина, що вказує кількість латентних змінних в цій суміші. $М$ - максимальна можлива кількість сумішей. \\
$\eta |_{\xi=i} \sim 1 + \text{Bin}(M-1; q_i)$ \\

$B_j^M$ - множина бінарних послідовностей з $j$ одиниць \\
$B^M = \cup_{j=1}^M B_j^M$ - множина всіх бінарних послідовностей \\
$S \in B^M: s=(s_1, \dots, s_M)$ \\
$S$ - випадкова величина на $B^M$

$$P(S=s| \xi=i, \eta=j) = \mathds{1}(s \in B^M_j) \cdot \frac{r_{i1}^{s_1} \cdots r_{iM}^{s_M}}{\sum_{s^{'} \in B^M_j} r_{i1}^{s^{'}_1} \cdots r_{iM}^{s^{'}_M}}$$


$\zeta$ - випадкова величина яка вказує до якої саме конкретної компоненти суміші з $j$ елементної множини одиниць $s$ належить $Y$.

$$P(\zeta=l | \xi=i, \eta=j, S=s) = \frac{s_l r_{il}}{\sum_{k=1}^{M} s_k \cdot r_{ik}}$$

Тоді:

$$(X, Y) = \sum_{i=1}^{K} \sum_{j=1}^{M} \sum_{s \in B^M_j} \sum_{\substack{l=1:\\ s_l=1}}^{M} \mathds{1}(\xi=i, \eta=j, S=s, \zeta=l) (X_i, Y_{il})$$

$$X_i \sim N(\mu_i, \sigma_i^2)$$
$$Y_{il} \sim N(\mu_{il}, \sigma_{il}^2)$$

$X_i$ та $Y_{il}$ при заданих $\xi, \eta, S, \zeta$ - незалежні

$$f_{(X, Y)}(x, y) = \sum_{i=1}^{K} \sum_{j=1}^{M} \sum_{s \in B^M_j} \sum_{\substack{l=1:\\ s_l=1}}^{M} P(\xi=i, \eta=j, S=s, \zeta=l) \cdot f_i(x; \theta_i) \cdot g_{il}(y; \theta_{il})$$

Якщо $(X_1, Y_1), \dots, (X_n, Y_n)$ - н.о.р. в.в. $\sim (X, Y)$ \\

$$L(x_1, \dots x_n; y_1, \dots y_n; P, Q, R, \Theta) = \prod_{t=1}^{n} \sum_{i=1}^{K} \sum_{j=1}^{M} \sum_{s \in B^M_j} \sum_{\substack{l=1:\\ s_l=1}}^{M} P(\xi^t=i, \eta^t=j, S^t=s, \zeta^t=l) \cdot f_i(x^t; \theta_i) \cdot g_{il}(y^t; \theta_{il})$$

\pagebreak
Введемо наступні позначення:\\
$$\pi_{ijsl} = P(\xi^t=i, \eta^t=j, S^t=s, \zeta^t=l)$$

$$\pi_{ijsl} = p_i \cdot q_{ij} \cdot \mathds{1}(s \in B^M_j) \cdot \frac{r_{i1}^{s_1} \cdots r_{iM}^{s_M}}{\sum_{s^{'} \in B^M_j} r_{i1}^{s^{'}_1} \cdots r_{iM}^{s^{'}_M}} \cdot \frac{s_l r_{il}}{\sum_{k=1}^{M} s_k \cdot r_{ik}} $$

$$z_{ijsl}^t = \mathds{1}(\xi^t=i, \eta^t=j, S^t=s, \zeta^t=l)$$

Тоді:

$$\log{L(x_1, \dots x_n; y_1, \dots y_n; P, Q, R, \Theta)} = \sum_{t=1}^{n} \sum_{i=1}^{K} \sum_{j=1}^{M} \sum_{s \in B^M_j} \sum_{\substack{l=1:\\ s_l=1}}^{M} z_{ijsl}^t \log{ \left(\pi_{ijsl} \cdot f_i(x^t; \theta_i) \cdot g_{il}(y^t; \theta_{il}) \right)}$$

Оцінюємо $z_{ijsl}^t$ як умовне математичне сподівання, тобто вводять нову змінну 
$$\gamma_{ijsl}^t = M[z_{ijsl}^t | x_1, \dots x_n; y_1, \dots y_n; P, Q, R, \Theta]$$

$$\gamma_{ijsl}^t = \frac{\pi_{ijsl} \cdot f_i(x^t; \theta_i) \cdot g_{il}(y^t; \theta_{il})}{\sum_{i'=1}^{K} \sum_{j'=1}^{M} \sum_{s' \in B^M_j} \sum_{\substack{l'=1:\\ {s'}_{l'}=1}}^{M} \pi_{i'j's'l'} \cdot f_{i'}(x^t; \theta_{i'}) \cdot g_{i'l}(y^t; \theta_{i'l'})}$$

Тоді квазі-логарифмічна функція набуває вигляду: \\
\begin{equation}
\begin{split}
QL(x_1, \dots x_n; y_1, \dots y_n; P, Q, R, \Theta) = \\
= \sum_{t=1}^{n} \sum_{i=1}^{K} \sum_{j=1}^{M} \sum_{s \in B^M_j} \sum_{\substack{l=1:\\ s_l=1}}^{M} \gamma_{ijsl}^t \cdot [\log{\pi_{ijsl}} + \log{f_i(x^t; \theta_i)} + \log{g_{il}(y^t; \theta_{il})}] = \\
= \sum_{t=1}^{n} \sum_{i=1}^{K} \sum_{j=1}^{M} \sum_{s \in B^M_j} \sum_{\substack{l=1:\\ s_l=1}}^{M} \gamma_{ijsl}^t \cdot [\log{p_i} + \log{q_{ij}} + \sum_{q=1}^{M} s_q \log{r_{iq}} + \log{f_i(x^t; \theta_i)} + \log{g_{il}(y^t; \theta_{il})}]
\end{split}
\end{equation}


\begin{align*}
& QL(x_1, \dots x_n; y_1, \dots y_n; P, Q, R, \Theta) = \\
& = \sum_{t=1}^{n} \sum_{i=1}^{K} \sum_{j=1}^{M} \sum_{s \in B^M_j} \sum_{\substack{l=1:\\ s_l=1}}^{M} \gamma_{ijsl}^t \cdot [\log{\pi_{ijsl}} + \log{f_i(x^t; \theta_i)} + \log{g_{il}(y^t; \theta_{il})}] = \\
& = \sum_{t=1}^{n} \sum_{i=1}^{K} \sum_{j=1}^{M} \sum_{s \in B^M_j} \sum_{\substack{l=1:\\ s_l=1}}^{M} \gamma_{ijsl}^t \cdot \Big[\log{p_i} + \log{q_{ij}} + \sum_{k=1}^{M} s_k \log{r_{ik}} -
\log{\left( \sum_{s^{'} \in B^M_j} r_{i1}^{s^{'}_1} \cdots r_{iM}^{s^{'}_M} \right)} +  \log{r_{il}}\\ 
& - \log{ \left(\sum_{k=1}^{M} s_k \cdot r_{ik} \right)} + \log{f_i(x^t; \theta_i)} + \log{g_{il}(y^t; \theta_{il})} \Big]
\end{align*}

\pagebreak

Обмеження: 
$$\sum_{j=1}^{M} r_{ij} = 1, \:\: i \in 1 \dots K$$
$$\sum_{j=1}^{M} q_{ij} = 1, \:\: i \in 1 \dots K$$
$$\sum_{i=1}^{K} p_{i} = 1$$

Сформуємо Лагранжіан:

\begin{align*}
& L(x_1, \dots x_n; y_1, \dots y_n; P, Q, R, \Theta) = \\
& = \sum_{t=1}^{n} \sum_{i=1}^{K} \sum_{j=1}^{M} \sum_{s \in B^M_j} \sum_{\substack{l=1:\\ s_l=1}}^{M} \gamma_{ijsl}^t \cdot \Big[\log{p_i} + \log{q_{ij}} + \sum_{k=1}^{M} s_k \log{r_{ik}} -
\log{\left( \sum_{s^{'} \in B^M_j} r_{i1}^{s^{'}_1} \cdots r_{iM}^{s^{'}_M} \right)} +  \log{r_{il}}\\ 
& - \log{ \left(\sum_{k=1}^{M} s_k \cdot r_{ik} \right)} + \log{f_i(x^t; \theta_i)} + \log{g_{il}(y^t; \theta_{il})} \Big] - \alpha \left( \sum_{i=1}^{K} p_i - 1 \right) - \sum_{i=1}^{K} \beta_i \left( \sum_{j=1}^{M} q_{ij} - 1 \right) - \\
& -  \sum_{i=1}^{K} \kappa_i \left( \sum_{j=1}^{M} r_{ij} - 1 \right)
\end{align*}

Залишилось знайти похідні:  


$$
\frac{\partial L}{\partial p_x} = \sum_{t=1}^{n} \sum_{j=1}^{M} \sum_{s \in B^M_j} \sum_{\substack{l=1:\\ s_l=1}}^{M} \gamma_{xjsl}^t \cdot \frac{1}{p_x} - \alpha = 0
$$

$$
p_x = \frac{1}{\alpha} \cdot \sum_{t=1}^{n} \sum_{j=1}^{M} \sum_{s \in B^M_j} \sum_{\substack{l=1:\\ s_l=1}}^{M} \gamma_{xjsl}^t 
$$

З іншого боку, $\sum_{x=1}^{K} p_x = 1$. Тому: 
$$\alpha = \sum_{i=1}^{K} \sum_{t=1}^{n} \sum_{j=1}^{M} \sum_{s \in B^M_j} \sum_{\substack{l=1:\\ s_l=1}}^{M} \gamma_{ijsl}^t = n$$

\begin{empheq}[box=\fbox]{align*}
p_x = \frac{1}{n} \cdot \sum_{t=1}^{n} \sum_{j=1}^{M} \sum_{s \in B^M_j} \sum_{\substack{l=1:\\ s_l=1}}^{M} \gamma_{xjsl}^t 
\end{empheq}

\pagebreak

$$
\frac{\partial L}{\partial q_{xy}} = \sum_{t=1}^{n} \sum_{s \in B^M_y} \sum_{\substack{l=1:\\ s_l=1}}^{M} \gamma_{xysl}^t \cdot \frac{1}{q_{xy}} - \beta_x = 0
$$

$$
q_{xy} = \frac{1}{\beta_x} \cdot \sum_{t=1}^{n} \sum_{s \in B^M_y} \sum_{\substack{l=1:\\ s_l=1}}^{M} \gamma_{xysl}^t 
$$

З іншого боку, $\sum_{y=1}^{M} q_{xy} = 1$. Тому: 
$$
\beta_x = \sum_{y=1}^{M} \sum_{t=1}^{n} \sum_{s \in B^M_y} \sum_{\substack{l=1:\\ s_l=1}}^{M} \gamma_{xysl}^t 
$$

\begin{empheq}[box=\fbox]{align*}
q_{xy} = \frac{\sum_{t=1}^{n} \sum_{s \in B^M_y} \sum_{\substack{l=1:\\ s_l=1}}^{M} \gamma_{xysl}^t}
{\sum_{t'=1}^{n} \sum_{y'=1}^{M} \sum_{s' \in B^M_{y'}} \sum_{\substack{l'=1:\\ {s'}_{l'}=1}}^{M} \gamma_{xy's'l'}^{t'}}
\end{empheq}

\vspace{1cm}


\begin{align*}
\frac{\partial L}{\partial r_{xy}} &= \sum_{t=1}^{n} \sum_{j=1}^{M} \sum_{s \in B^M_j} \frac{\partial}{\partial r_{xy}} \sum_{i=1}^{K}  \sum_{\substack{l=1:\\ s_l=1}}^{M} \gamma_{ijsl}^t \Big[\sum_{k=1}^{M} s_k \log{r_{ik}} + \log{r_{il}} \Big] - \\
& - \sum_{t=1}^{n} \sum_{j=1}^{M} \sum_{s \in B^M_j} \frac{\partial}{\partial r_{xy}} \sum_{i=1}^{K}  \sum_{\substack{l=1:\\ s_l=1}}^{M} \gamma_{ijsl}^t \log{\left( \sum_{s^{'} \in B^M_j} r_{i1}^{s^{'}_1} \cdots r_{iM}^{s^{'}_M} \right)} - \\ 
& - \sum_{t=1}^{n} \sum_{j=1}^{M} \sum_{s \in B^M_j} \frac{\partial}{\partial r_{xy}} \sum_{i=1}^{K}  \sum_{\substack{l=1:\\ s_l=1}}^{M} \gamma_{ijsl}^t \log{ \left(\sum_{k=1}^{M} s_k \cdot r_{ik} \right)} - \kappa_x = 0
\end{align*}

Поки відкинемо останні два доданки з похідними і виразимо параметр $r_{xy}$:
$$ \frac{\partial L}{\partial r_{xy}} = \sum_{t=1}^{n} \sum_{j=1}^{M} \sum_{s \in B^M_j} \frac{\partial}{\partial r_{xy}} \sum_{i=1}^{K}  \sum_{\substack{l=1:\\ s_l=1}}^{M} \gamma_{ijsl}^t \Big[\sum_{k=1}^{M} s_k \log{r_{ik}} + \log{r_{il}} \Big] $$ 

$$
\sum_{t=1}^{n} \sum_{j=1}^{M} \sum_{s \in B^M_j} 
\frac{\gamma_{xjsy}^t }{r_{xy}} + 
\frac{s_y}{r_{xy}}\sum_{\substack{l=1:\\ s_l=1}}^{M} \gamma_{xjsl}^t - \kappa_x = 0
$$

Звідси:

$$
r_{xy} = \frac{1}{\kappa_x} \sum_{t=1}^{n} \sum_{j=1}^{M} \sum_{s \in B^M_j} 
\Big[ \gamma_{xjsy}^t + 
s_y \sum_{\substack{l=1:\\ s_l=1}}^{M} \gamma_{xjsl}^t \Big]
$$

\pagebreak
Відомо, що:
$$
\sum_{y=1}^{M} r_{xy} = 1
$$

Тому: 
$$\kappa_x = \sum_{y=1}^{M} \sum_{t=1}^{n} \sum_{j=1}^{M} \sum_{s \in B^M_j} 
\Big[ \gamma_{xjsy}^t + 
s_y \sum_{\substack{l=1:\\ s_l=1}}^{M} \gamma_{xjsl}^t \Big]
$$

\begin{empheq}[box=\fbox]{align*}
r_{xy} = \frac{\sum_{t=1}^{n} \sum_{j=1}^{M} \sum_{s \in B^M_j} 
\Big[ \gamma_{xjsy}^t + 
s_y \sum_{\substack{l=1:\\ s_l=1}}^{M} \gamma_{xjsl}^t \Big]}
{\sum_{y=1}^{M} \sum_{t=1}^{n} \sum_{j=1}^{M} \sum_{s \in B^M_j} 
\Big[ \gamma_{xjsy}^t + 
s_y \sum_{\substack{l=1:\\ s_l=1}}^{M} \gamma_{xjsl}^t \Big]}
\end{empheq}


\end{document}